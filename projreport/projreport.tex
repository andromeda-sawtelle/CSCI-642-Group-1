%% The first command in your LaTeX source must be the \documentclass command.
\documentclass[sigconf, anonymous]{acmart}

\def\BibTeX{\textsc{Bib}\TeX}
\def\LaTeX{\textsc{La}\TeX}
\usepackage{url}
\usepackage{balance}
\usepackage{color}
\usepackage{enumerate}
\settopmatter{printacmref=false}

\setcopyright{none}
\renewcommand\footnotetextcopyrightpermission[1]{}
\pagestyle{plain}
\acmConference[CSCI]{}{Project Report}{2024}

\begin{document}
\title{Investigating the Security of the Smaller C Compiler}

\author{Anonymous}
\affiliation{
  \institution{Removed}
}

\begin{abstract}
  The abstract is a fancy way to saying {\bf Summary.} It should
  preferably be two paragraphs that summarize your report. Abstracts
  are read independently from the rest of the report so you must not
  cite your report or any other papers. Study other abstracts in the
  papers you have been reading to understand what an abstract should
  mean, although many are poorly written.

  The abstract is not an introduction or overview of your report! It is
  a summary of your report, which should include the background,
  context, content, and contributions/results of your report. Make
  sure you only take credit for what you did (which is the comparison
  and your learning) and mention all work (research ideas, software,
  etc.) done by others.

  The first paragraph should provide an overview of the topics being
  covered in the report. The second paragraph should describe what you
  learned and and how it would be meaningful to your reader, but do
  not this in the first person.
  
  {\bf Write the entire abstract in the third person and in past
    tense. It should typically be around 200-250 words.}
\end{abstract}

\keywords{Come up with your own descriptive keywords to make possible
  for a potential reader to find your report.}

\maketitle


\section{Overview}
\label{motivation}

We are investigating how the “Smaller C” compiler addresses various security vulnerabilities common in compilers, examining the security impact of any bugs or compiler restrictions, and documenting any vulnerabilities that we find.

Some compilers are able to catch and address many of the most common software vulnerabilities described in the Common Weakness Enumeration.\cite{CWE:2023} Of these vulnerabilities, only a few can theoretically be caught by compilers.

The first of these bugs is Out-of-bounds read and writes, which occur when memory outside of an allocated space is accessed in the program. This can lead to undefined behavior, and can be exploited by malicious actors to change program functionality. Ideally, static out-of-bounds reference should be caught by the compiler and flagged as an error, but dynamic accesses can still pass through.

A second issue is use after free (UAF), in which memory that was allocated for use is freed, then accessed later. Again, this can be used to cause unintended behavior if exploited. In sequential programs, it should be possible to prevent this, but it would also be ideal to ensure that memory is freed in the same scope it is declared, and only after it has finished being used. Most C compilers do not address this, but using an address sanitation procedure to detect memory errors in implementation can reduce this vulnerability. Another possible vulnerability is null-pointer dereferencing. Null-pointer dereferencing refers to attempting to read memory from a null-pointer, which could be a pointer that has not yet been assigned, or a pointer that was freed and reset. Checks like those described above could help with this, but these tasks will likely be harder, due to the single-pass nature of Smaller C. 

A particularly difficult to diagnose vulnerability that could be addressed is the use of shared resources with improper synchronization. This refers to accessing any shared data or resource in a multi-threaded system in a way that either interferes with other programs trying to use the same resource, or prevents other programs from being able to access that resource. If a resource is declared as “critical,” which means that it should not be accessed by multiple sources during an operation, it should always be surrounded by a semaphore, lock, or other equivalent security measure that ensures exclusive access to it. A compiler could check for the entering and exiting of critical sections when accessing critical resources, so we will see if Smaller C implements any such behavior. 

A vulnerability that C compilers often have is one where stack data can be overwritten by invalid memory accesses, overflows, or other common code vulnerabilities. In the worst case, overwriting data on the stack can allow the function return address to be overwritten, allowing malicious actors to execute arbitrary code by inserting a faulty address. It will be important to investigate how Smaller C attempts to prevent these types of attacks. David Monniaux details one method for catching these attacks, in which a randomly generated “canary” value is inserted before the location that the return address is stored.~\cite{Monniaux:Memory} This value is checked at the end of function execution. If it was changed, then much like a canary in a coal mine, it signifies that something has gone wrong. In this case, it means that the return address was likely changed too, so the program stops execution to prevent any undesired behavior. This method may not even be possible in a single-pass compiler, so some other work-around may be necessary.

Code Optimization in compilers can often unintentionally remove security checks or make code less secure. This can be especially true when compiling on hardware different than the intended program location. We will explore how Smaller C handles common optimization techniques, such as dead store elimination. D’Silva et al. laid out how dead store elimination in GCC removed a necessary security feature from the function crypt().~\cite{DSilva:Correctness} We will see if Smaller C has similar erroneous optimization. Another common error with optimization is how the compiler handles undefined behavior within the C standard. There have been noted cases, by Wang et al. where various security checks are removed due to it being undefined by the C standard.~\cite{Wang:Optimization}

Smaller C has some unique limitations that may, if improperly implemented, lead to security vulnerabilities. First, there is no stack overflow check, and very few compile-time checks compared to most multi-pass compilers. This might cause a program to be able to be frozen by putting it into an infinite loop. Secondly, large integers constant are ambiguously assigned to types of different sizes unless they are explicitly marked. This can cause different amounts of space to be allocated, which can cause issues in extreme cases.

Over the course of this project, we intend to document how Smaller C addresses these vulnerabilities, and provide solutions or improvements. It may not be possible to prevent these issues, but it should be possible to add safety features to catch these issues most of the time.

\section{Related Work}
\label{related work}

Review the related work in the literature, both research and practice,
to place your project in perspective, and what other people have been
doing to address this problem. Make sure your literature survey is
fairly complete and recent, which means you should references that are
as new as possible, i.e., some (not necessarily all) should be in the
current or last year.

You must cite your sources correctly per ACM style guidelines (and of
course, you need to use \LaTeX and \BibTeX correctly).

\section{Design and Implementation}
\label{design}

Use this section to describe the basic design, architecture, and
implementation of your project

\section{Analysis}
\label{analysis}
Use this section to describe the analysis of your project that you
conducted, and whether the results are meaningful or not.

Also, discuss what all you learned from the project, especially what
mistakes to avoid in the future.

\section{Legal Considerations}
\label{legal considerations}

Use this section to discuss legal issues relevant to your project,
especially relating aspects of data that are relevant to your project.

Use the textbook and your readings to guide the legal aspects of your
discussion. Look at the laws that have been passed in recent years,
and look at legislation that is being proposed in the space covered by
your project.

\section{Ethical Considerations}
\label{ethical considerations}

Use this section to discuss ethical issues relevant to your
project, especially relating aspects of data that are relevant to your
project.

Use the ACM Code to guide the ethical aspects of your
discussion~\cite{ACMCODE}.


\section{Conclusions}
\label{conclusions}

Use this section to describe the current status of your work
and what else needs to be done.

Also, discuss what further directions your work can be taken by
others.

Finally, present some final words to place your project in
perspective.

\balance

%%%%%%%%%%%%%%%%%%%%%%%%%%%%%%%%%%%%%%%%%%%%%%%%%
%%% DELETE FROM HERE 
%%%%%%%%%%%%%%%%%%%%%%%%%%%%%%%%%%%%%%%%%%%%%%%%%
\begin{comment}
\section*{Tables, Figures, and Citations/References -
  DELETE THIS SUBSECTION BEFORE ANY SUBMISSION}

{\bf This unnumbered section is meant to provide you with some help in
  dealing with figures, tables and citations, as these are sometimes
  hard for people new to \LaTeX. Your figures, tables and citations
  must be distributed all over your report (not here), as appropriate
  for your report. So here is a quick guide extracted from the ACM
  style guide.

  Please delete this entire section before you submit! If
  I see this section in your report, you will lose points!!!}

Tables, figures, and citations/references in technical documents need
to be presented correctly. In proper technical English writing (for
reasons beyond the scope of this discussion), table captions are above
the table and figure captions are below the figure.

Figures in your report must be original, that is, created by
the student to reflect your analysis and understanding: please do not
screen-scrape and cut-and-paste figures from any other paper you have
read. If you need to, just cite the figure in the original paper and
summarize what points you want to make in this report.

\begin{figure}[tb]
\begin{center}
\includegraphics[width=1.5in]{rit-tiger-with-text.jpg}
\caption{The cutest tiger in the world (JPG).}
\label{fig:SAMPLE FIGURE}
\end{center}
\end{figure}

\begin{figure*}[tb]
\begin{center}
\includegraphics[width=4.5in]{rit-tiger-with-text.jpg}
\caption{The fiercest tiger in the world (JPG).}
\label{fig:BIG SAMPLE FIGURE}
\end{center}
\end{figure*}

When you need to cite any original figures in your own report, they
should be handled as demonstrated here. State that Figure~\ref{fig:SAMPLE
  FIGURE} is a simple illustration used in the ACM Style sample
document. Again, never refer to the figure below (or above) because
figures may be placed by \LaTeX{} at any appropriate location that can
change when you recompile your source $.tex$ file. Also, if you need a
figure to be legible, you may want it to span both columns. For
example, the same tiger can be scaled up as shown in
Figure~\ref{fig:BIG SAMPLE FIGURE} to span both columns.

\begin{table}[tb]
\centering
\caption{Issue Resolution}
\label{tab:SAMPLE TABLE}
\begin{tabular}{|l|r|l|} \hline
Issue&Percentage&Assignment Summary\\ \hline
Issue 1 &  5\% & Best programmers\\ \hline
Issue 2 &  30\% &New full-time hires\\ \hline
Issue 3 &  70\% & New co-op students on this\\ \hline
Issue 4 &  90\% & Keep on back-burner for now\\ \hline
\end{tabular}
\end{table}

Issues in this sample report, as shown in Table~\ref{tab:SAMPLE
  TABLE}. Note that tables or figures are never stated as being above
or below, as the typesetting is at liberty to place them anywhere
meaningful

Finally, citing documents needs to be done properly too. For example,
Bowman, Debray, and Peterson~\cite{bowman:reasoning} reason about
different naming systems. One of the common types of citations these
days is to items only posted on the Web such as this 2014 CMU SEI
webinar by Dormann et al.~\cite{dormann:API}.

You will find the \BibTeX{} entries needed for many papers that are
being cited at the ACM or IEEE digital libraries, or other sources on
the web, otherwise you can write your own versions easily and add them
to the $*.bib$ file in the folder. There are many sample bibtex
template files that can be used to model your own references. Please
refer to the instructor's papers for guidance.

The list of all references will be generated in the standard ACM Reference
style using \LaTeX{}/\BibTeX{} correctly. Note that you
need to first the following sequence to get the report
compiled correctly:

\begin{enumerate}
\item {\tt latex} {\em projreport}
\item {\tt bibtex} {\em projreport}
\item {\tt latex} {\em projreport}
\item {\tt latex} {\em projreport}
\end{enumerate}
\end{comment}
%%%%%%%%%%%%%%%%%%%%%%%%%%%%%%%%%%%%%%%%%%%%%%%%%
%%% DELETE UNTIL HERE
%%%%%%%%%%%%%%%%%%%%%%%%%%%%%%%%%%%%%%%%%%%%%%%%%



\bibliographystyle{ACM-Reference-Format}
\bibliography{projreport} 

\end{document}
