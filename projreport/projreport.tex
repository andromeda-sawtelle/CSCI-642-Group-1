%% The first command in your LaTeX source must be the \documentclass command.
\documentclass[sigconf, anonymous]{acmart}

\def\BibTeX{\textsc{Bib}\TeX}
\def\LaTeX{\textsc{La}\TeX}
\usepackage{url}
\usepackage{balance}
\usepackage{color}
\usepackage{enumerate}
\settopmatter{printacmref=false}

\setcopyright{none}
\renewcommand\footnotetextcopyrightpermission[1]{}
\pagestyle{plain}
\acmConference[CSCI]{}{Project Report}{2024}

\begin{document}
\title{Project Report (Rename Appropriately)}

\author{Anonymous}
\affiliation{
  \institution{Removed}
}

\begin{abstract}
  The abstract is a fancy way to saying {\bf Summary.} It should
  preferably be two paragraphs that summarize your report. Abstracts
  are read independently from the rest of the report so you must not
  cite your report or any other papers. Study other abstracts in the
  papers you have been reading to understand what an abstract should
  mean, although many are poorly written.

  The abstract is not an introduction or overview of your report! It is
  a summary of your report, which should include the background,
  context, content, and contributions/results of your report. Make
  sure you only take credit for what you did (which is the comparison
  and your learning) and mention all work (research ideas, software,
  etc.) done by others.

  The first paragraph should provide an overview of the topics being
  covered in the report. The second paragraph should describe what you
  learned and and how it would be meaningful to your reader, but do
  not this in the first person.
  
  {\bf Write the entire abstract in the third person and in past
    tense. It should typically be around 200-250 words.}
\end{abstract}

\keywords{Come up with your own descriptive keywords to make possible
  for a potential reader to find your report.}

\maketitle


\section{Overview}
\label{motivation}

Use three or four paragraphs to present an overview of your project
(in less than a page), its background, motivate your work, citing
references here, and lay out the main goals of your project, briefly
describing them. Be clear.

Provide a roadmap for the remaining sections of the paper. For
example, you can state something like this:

This report is organized as follows. Section~\ref{related work}
discusses work done by others in this area. Section~\ref{design}
presents a discussion of the design, architecture, and implementation
issues we considered, along with any special aspects of our
project. Section~\ref{analysis} presents an analysis of the project,
along with lessons learned.  Section~\ref{legal considerations}
discusses the legal considerations of the issues relevant to our
project and section~\ref{ethical considerations} discusses the ethical
considerations of the issues relevant to our
project. Section~\ref{conclusions} presents the current state of the
project, possible future work, and then concludes with a few final
remarks.

{\bf Note: This specific file is a generic template with some sections
  may not be required. If you think section titles do not fit your
  specific needs, set up an appointment with the instructor to get
  permission before changing them.}


\section{Related Work}
\label{related work}

Review the related work in the literature, both research and practice,
to place your project in perspective, and what other people have been
doing to address this problem. Make sure your literature survey is
fairly complete and recent, which means you should references that are
as new as possible, i.e., some (not necessarily all) should be in the
current or last year.

You must cite your sources correctly per ACM style guidelines (and of
course, you need to use \LaTeX and \BibTeX correctly).

\section{Design and Implementation}
\label{design}

Use this section to describe the basic design, architecture, and
implementation of your project

\section{Analysis}
\label{analysis}
Use this section to describe the analysis of your project that you
conducted, and whether the results are meaningful or not.

Also, discuss what all you learned from the project, especially what
mistakes to avoid in the future.

\section{Legal Considerations}
\label{legal considerations}

Use this section to discuss legal issues relevant to your project,
especially relating aspects of data that are relevant to your project.

Use the textbook and your readings to guide the legal aspects of your
discussion. Look at the laws that have been passed in recent years,
and look at legislation that is being proposed in the space covered by
your project.

\section{Ethical Considerations}
\label{ethical considerations}

Use this section to discuss ethical issues relevant to your
project, especially relating aspects of data that are relevant to your
project.

Use the ACM Code to guide the ethical aspects of your
discussion~\cite{ACMCODE}.


\section{Conclusions}
\label{conclusions}

Use this section to describe the current status of your work
and what else needs to be done.

Also, discuss what further directions your work can be taken by
others.

Finally, present some final words to place your project in
perspective.

\balance

%%%%%%%%%%%%%%%%%%%%%%%%%%%%%%%%%%%%%%%%%%%%%%%%%
%%% DELETE FROM HERE 
%%%%%%%%%%%%%%%%%%%%%%%%%%%%%%%%%%%%%%%%%%%%%%%%%

\section*{Tables, Figures, and Citations/References -
  DELETE THIS SUBSECTION BEFORE ANY SUBMISSION}

{\bf This unnumbered section is meant to provide you with some help in
  dealing with figures, tables and citations, as these are sometimes
  hard for people new to \LaTeX. Your figures, tables and citations
  must be distributed all over your report (not here), as appropriate
  for your report. So here is a quick guide extracted from the ACM
  style guide.

  Please delete this entire section before you submit! If
  I see this section in your report, you will lose points!!!}

Tables, figures, and citations/references in technical documents need
to be presented correctly. In proper technical English writing (for
reasons beyond the scope of this discussion), table captions are above
the table and figure captions are below the figure.

Figures in your report must be original, that is, created by
the student to reflect your analysis and understanding: please do not
screen-scrape and cut-and-paste figures from any other paper you have
read. If you need to, just cite the figure in the original paper and
summarize what points you want to make in this report.

\begin{figure}[tb]
\begin{center}
\includegraphics[width=1.5in]{rit-tiger-with-text.jpg}
\caption{The cutest tiger in the world (JPG).}
\label{fig:SAMPLE FIGURE}
\end{center}
\end{figure}

\begin{figure*}[tb]
\begin{center}
\includegraphics[width=4.5in]{rit-tiger-with-text.jpg}
\caption{The fiercest tiger in the world (JPG).}
\label{fig:BIG SAMPLE FIGURE}
\end{center}
\end{figure*}

When you need to cite any original figures in your own report, they
should be handled as demonstrated here. State that Figure~\ref{fig:SAMPLE
  FIGURE} is a simple illustration used in the ACM Style sample
document. Again, never refer to the figure below (or above) because
figures may be placed by \LaTeX{} at any appropriate location that can
change when you recompile your source $.tex$ file. Also, if you need a
figure to be legible, you may want it to span both columns. For
example, the same tiger can be scaled up as shown in
Figure~\ref{fig:BIG SAMPLE FIGURE} to span both columns.

\begin{table}[tb]
\centering
\caption{Issue Resolution}
\label{tab:SAMPLE TABLE}
\begin{tabular}{|l|r|l|} \hline
Issue&Percentage&Assignment Summary\\ \hline
Issue 1 &  5\% & Best programmers\\ \hline
Issue 2 &  30\% &New full-time hires\\ \hline
Issue 3 &  70\% & New co-op students on this\\ \hline
Issue 4 &  90\% & Keep on back-burner for now\\ \hline
\end{tabular}
\end{table}

Issues in this sample report, as shown in Table~\ref{tab:SAMPLE
  TABLE}. Note that tables or figures are never stated as being above
or below, as the typesetting is at liberty to place them anywhere
meaningful

Finally, citing documents needs to be done properly too. For example,
Bowman, Debray, and Peterson~\cite{bowman:reasoning} reason about
different naming systems. One of the common types of citations these
days is to items only posted on the Web such as this 2014 CMU SEI
webinar by Dormann et al.~\cite{dormann:API}.

You will find the \BibTeX{} entries needed for many papers that are
being cited at the ACM or IEEE digital libraries, or other sources on
the web, otherwise you can write your own versions easily and add them
to the $*.bib$ file in the folder. There are many sample bibtex
template files that can be used to model your own references. Please
refer to the instructor's papers for guidance.

The list of all references will be generated in the standard ACM Reference
style using \LaTeX{}/\BibTeX{} correctly. Note that you
need to first the following sequence to get the report
compiled correctly:

\begin{enumerate}
\item {\tt latex} {\em projreport}
\item {\tt bibtex} {\em projreport}
\item {\tt latex} {\em projreport}
\item {\tt latex} {\em projreport}
\end{enumerate}

%%%%%%%%%%%%%%%%%%%%%%%%%%%%%%%%%%%%%%%%%%%%%%%%%
%%% DELETE UNTIL HERE
%%%%%%%%%%%%%%%%%%%%%%%%%%%%%%%%%%%%%%%%%%%%%%%%%



\bibliographystyle{ACM-Reference-Format}
\bibliography{projreport} 

\end{document}
